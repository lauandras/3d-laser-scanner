\documentclass[12pt,a4paper]{article}
\usepackage[utf8]{inputenc}
\usepackage{pdfpages}
\usepackage[magyar]{babel}
\usepackage{hyperref}	
\hypersetup{					
colorlinks=false,						
pdfborder={0 0 0},
}
\usepackage{fancyhdr}
\usepackage[left=2cm,right=2cm,top=3.5cm,bottom=3cm,headsep=50pt]{geometry}

\begin{document}
%=================================================
\thispagestyle{empty}
\begin{center}
\includegraphics[scale=0.3]{bme.pdf}\\
\large{Budapesti Műszaki- és Gazdaságtudományi Egyetem\\
Gépészmérnöki kar}\\[1cm]
\begin{Huge}
\textbf{Mechatronika projekt}
\end{Huge}\\[0.5cm]

\Large{BMEGEFOAMM3}\\[2cm]
\Huge{\bf{3D szkenner}}\\[2.5cm]
\bf{\Large{Tar Dániel\\[5pt]
		Bognár Máté\\
		Varga Roland}}\\[1cm]

\includegraphics[scale=0.5]{mogilogo.jpg}\\[1cm]
\large{\today}
\end{center}
%=================================================
\newpage
\tableofcontents
\newpage
\pagestyle{fancy}
\fancyhf{}
\rhead{Tar Dániel\\Bognár Máté\\Varga Roland}
\lhead{3D szkenner}
\cfoot{\thepage. oldal}
%=================================================
\section{Feladat leírása}
%=================================================
\section{Alapul vett szakirodalom}
Ide jönne a két nagyon hasonló projekt meghivatkozva!



Roland help a hivatkozassal :D
%=================================================
\section{Felhasznált hardverek}
\subsection{Beszerzett eszközök}
\subsection{Saját készítésű eszközök}
\subsection{Hardverek közötti csatlakozás}
%=================================================
\section{Programkód}
Az egyes szorosan összetartozó részeket külön függvényben írtuk meg, amiket a főprogram, a scan\_object.m hív meg. A függvények csak a továbbiakban is használt változókat adják vissza, így redukáltuk a Workspace-en található vektorok számát.
\subsection{Kamera kalibráció}
\subsection{Perifériák inicializálása}
\subsection{Képek vágása}
\subsection{Transzformáció meghatározása}
\subsection{Szükséges változók deklarálása}
\subsection{Szkennelés folyamata}
\subsubsection{Lézerfény detektálása}
\subsubsection{A forgóasztal léptetése}
\subsubsection{Képek transzformációja}
\subsubsection{Pontfelhő generálása}
%=================================================
\section{Eredmények, a módszer korlátai}
%=================================================
\section{Továbbfejlesztési irányok}
%=================================================
\newpage
\begin{thebibliography}{9} 
\bibitem{feladatkiiras} 
Feladatkiírás\\
\texttt{http://mogi.bme.hu/letoltes/MECHATRONIKAI\%20\&\%20IR\%C3\%81NY\%C3\%8DT\%C3\%81\\
STECHNIKAI\%20T\%C3\%81RGYAK/MECHATRONIKA\_PROJEKT\_BMEGEFOAMM3/Feladatlapok\_M/}\\
2018.05.16.
\end{thebibliography}

\end{document}